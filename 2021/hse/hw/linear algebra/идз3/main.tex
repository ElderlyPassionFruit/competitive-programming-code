\documentclass[12pt]{article}

\usepackage[T2A]{fontenc}
\usepackage[unicode=true, colorlinks=true, linkcolor=blue, urlcolor=blue]{hyperref}
\usepackage{nameref}
\usepackage{amsmath}
\usepackage{amssymb}
\usepackage{dsfont}
\usepackage[russian]{babel}
\usepackage[margin=15mm]{geometry}
\usepackage{ragged2e}
\usepackage{nicefrac}
\usepackage{array}

\newcommand{\elon}[3]{%
  \ensuremath{\text{Э}_1(#1,\; #2,\; #3)}%
}
\newcommand{\eltw}[2]{%
  \ensuremath{\text{Э}_2(#1,\; #2)}%
}
\newcommand{\elth}[2]{%
  \ensuremath{\text{Э}_3(#1,\; #2)}%
}
 
\newcommand{\eqon}[3]{%
  \ensuremath{\overset{\text{Э}_1(#1,\; #2,\; #3)}{=}}%
}
\newcommand{\eqtw}[2]{%
  \ensuremath{\overset{\text{Э}_2(#1,\; #2)}{=}}%
}
\newcommand{\eqth}[2]{%
  \ensuremath{\overset{\text{Э}_3(#1,\; #2)}{=}}%
}

\newcommand{\arron}[3]{%
  \ensuremath{\xrightarrow{\text{Э}_1(#1,\; #2,\; #3)}}%
}
\newcommand{\arrtw}[2]{%
  \ensuremath{\xrightarrow{\text{Э}_2(#1,\; #2)}}%
}
\newcommand{\arrth}[2]{%
  \ensuremath{\xrightarrow{\text{Э}_3(#1,\; #2)}}%
}


\title{ИДЗ-3\\Вариант \textnumero 20}
\author{    
    Игорь Маркелов 203 группа \\ \href{https://t.me/ElderlyPassionFruit}{Telegram}
}

\begin{document}
    \maketitle
    \section*{Задача \textnumero 1}
        Для решения данной чудесной задачи мне нужен гениальный алгоритм от Дмитрия Витальевича Трушина 
        по разложению матрицы $A$ в $rk(A)$ матриц $x_i, {rk(x_i)=1}$.
        В начале рассмотрим подпространство в $\mathds{F}^5$ натянутое на вектор-стобцы данной матрицы. Выделим среди данных столбцов базис, и выразим оставшиеся (зависимые) столбцы через базисные. (Видимо нужно сказать магические слова про алгоритм с семинаров)
        \\\\
        $\left(\begin{array}{rrrrr}
       -1 & 5 & -16 & -5 & 11\\
        1 & 1 & 3 & 5 & -3\\
        3 & 5 & 6 & -1 & -1\\
        1 & -7 & 21 & -3 & -11\\
        2 & 6 & -2 & 2 & 2\\
        \end{array}\right) \arrth{5}{\frac{1}{2}}
        \left(\begin{array}{rrrrr}
        -1 & 5 & -16 & -5 & 11\\
        1 & 1 & 3 & 5 & -3\\
        3 & 5 & 6 & -1 & -1\\
        1 & -7 & 21 & -3 & -11\\
        1 & 3 & -1 & 1 & 1\\            
        \end{array}\right) \arrtw{1}{5} 
        \\\\\\
        \left(\begin{array}{rrrrr}
        1 & 3 & -1 & 1 & 1\\
        1 & 1 & 3 & 5 & -3\\
        3 & 5 & 6 & -1 & -1\\
        1 & -7 & 21 & -3 & -11\\
        -1 & 5 & -16 & -5 & 11\\            
        \end{array}\right) \overset{\elon{2}{1}{-1}}{\underset{\elon{3}{1}{-3}}{\underset{\elon{4}{1}{-1}}{\arron{5}{1}{1}}}}
        \left(\begin{array}{rrrrr}
        1 & 3 & -1 & 1 & 1\\
        0 & -2 & 4 & 4 & -4\\
        0 & -4 & 9 & -4 & -4\\
        0 & -10 & 22 & -4 & -12\\
        0 & 8 & -17 & -4 & 12\\           
        \end{array}\right) \overset{\elth{2}{-\frac{1}{2}}}{\arrth{4}{-\frac{1}{2}}}
        \\\\\\
        \left(\begin{array}{rrrrr}      
        1 & 3 & -1 & 1 & 1\\
        0 & 1 & -2 & -2 & 2\\
        0 & -4 & 9 & -4 & -4\\
        0 & 5 & -11 & 2 & 6\\
        0 & 8 & -17 & -4 & 12\\
        \end{array}\right) \overset{\elon{3}{2}{4}}{\underset{\elon{4}{2}{-5}}{\arron{5}{2}{-8}}} 
        \left(\begin{array}{rrrrr}
        1 & 3 & -1 & 1 & 1\\
        0 & 1 & -2 & -2 & 2\\
        0 & 0 & 1 & -12 & 4\\
        0 & 0 & -1 & 12 & -4\\
        0 & 0 & -1 & 12 & -4\\
        \end{array}\right) \overset{\elon{4}{3}{1}}{\arron{5}{3}{1}} 
        \\\\\\
        \left(\begin{array}{rrrrr}
        1 & 3 & -1 & 1 & 1\\
        0 & 1 & -2 & -2 & 2\\
        0 & 0 & 1 & -12 & 4\\
        0 & 0 & 0 & 0 & 0\\
        0 & 0 & 0 & 0 & 0\\
        \end{array}\right) \overset{\elon{2}{3}{2}}{\arron{1}{3}{1}} 
        \left(\begin{array}{rrrrr}
        1 & 3 & 0 & -11 & 5\\
        0 & 1 & 0 & -26 & 10\\
        0 & 0 & 1 & -12 & 4\\
        0 & 0 & 0 & 0 & 0\\
        0 & 0 & 0 & 0 & 0\\
        \end{array}\right) \arron{1}{2}{-3} 
        \left(\begin{array}{rrrrr}
        1 & 0 & 0 &  67 & -25\\
        0 & 1 & 0 & -26 &  10\\
        0 & 0 & 1 & -12 &   4\\
        0 & 0 & 0 &   0 &   0\\
        0 & 0 & 0 &   0 &   0\\
        \end{array}\right)$
        \\\\\\\\\\\\
        Далее базисные векторы - это вектор-столбцы с главными переменными, то есть $A^{(1)}, A^{(2)}, A^{(3)}$,
        а зависимые - $A^{(4)} \text{ и } A^{(5)}$. \\
        $A^{(4)}=67 \cdot A^{(1)} - 26 \cdot A^{(2)} - 12 \cdot A^{(3)}$, $A^{(5)}=-25 \cdot A^{(1)} + 10 \cdot A^{(2)} +4 \cdot A^{(3)}$, следовательно в качестве ответа можно взять следующие матрицы:
        \\\\
        $x_1 = \left(\begin{array}{r|r|r|r|r}
        A^{(1)} & 0 & 0 & 67 \cdot A^{(1)} & -25 \cdot A^{(1)}
        \end{array}\right),
        \\ 
        x_2 = \left(\begin{array}{r|r|r|r|r}
        0 & A^{(2)} & 0 & -26 \cdot A^{(2)} & 10 \cdot A^{(2)}
        \end{array}\right),
        \\
        x_3 = \left(\begin{array}{r|r|r|r|r}
        0 & 0 & A^{(3)} & -12 \cdot A^{(3)} & 4 \cdot A^{(3)}
        \end{array}\right)$
        \\\\
        \text{Они же, но в числах:}
        \\\\
         $x_1 =  \left(\begin{array}{rrrrr}
        -1 & 0 & 0 & -67 &   25\\
         1 & 0 & 0 &  67 &  -25\\
         3 & 0 & 0 & 201 &  -75\\
         1 & 0 & 0 &  67 &  -25\\
         2 & 0 & 0 & 134 &  -50
        \end{array}\right),
        x_2 =  \left(\begin{array}{rrrrr}
         0 & 5  & 0 & -130 &  50\\
         0 & 1  & 0 &  -26 &  10\\
         0 & 5  & 0 & -130 &  50\\
         0 & -7 & 0 &  182 & -70\\
         0 & 6  & 0 & -156 &  60
        \end{array}\right),
        x_3 =  \left(\begin{array}{rrrrr}
         0 & 0 & -16 &  192 & -64\\
         0 & 0 &   3 &  -36 &  12\\
         0 & 0 &   6 &  -72 &  24\\
         0 & 0 &  21 & -252 &  84\\
         0 & 0 &  -2 &   24 &  -8
        \end{array}\right)$
        \\\\
        P.S. $rk(A)=3$, т.к. количество независимых столбцов -- $3$, следовательно нужно разложить в $3$ матрицы с $rk(x_i)=1$, я это и сделал, во всех трёх матрицах: $x_1, x_2, x_3$, все столбцы выражаются через какой-то (и он не нулевой), следовательно их ранги равны $1$, следовательно мой ответ не лажа, а то, что нужно в задаче.
        \\
        Это первый раз в моей жизни когда я чёта техал, так что выглядит просто отвратно. $-6.5$ часов жизни тупа. Если будет не $1$ за данную задачу я прыгну в окно.  


    \section*{Задача \textnumero 2}
        Для начала найдём какой-нибудь базис $U$. Уложим векторы $u_i$ в матрицу по строчкам и применим алгоритм Гаусса. (Да, это тоже стандартный алгоритм с семинаров)
        \\\\
        $\left(\begin{array}{rrrr}
        3 & -2 & -3 & 2\\
        -13 & 9 & 8 & -5\\
        9 & -7 & 6 & -5\\
        6 & -4 & 5 & -3\\
        \end{array}\right) \overset{\elon{3}{1}{-3}}{\underset{\elth{2}{3}}{\arron{4}{1}{-2}}}
        \left(\begin{array}{rrrr}
        3 & -2 & -3 & 2\\
        -39 & 27 & 24 & -15\\
        0 & -1 & 15 & -11\\
        0 & 0 & 11 & 1\\
        \end{array}\right) \arron{2}{1}{13}
        \left(\begin{array}{rrrr}
        3 & -2 & -3 &   2\\
        0 &  1 &-15 &  11\\
        0 & -1 & 15 & -11\\
        0 &  0 & 11 & 1\\
        \end{array}\right) \arron{3}{2}{1}
        \\\\
        \left(\begin{array}{rrrr}
        3 & -2 & -3 &   2\\
        0 &  1 &-15 &  11\\
        0 &  0 & 11 & 1\\
        \end{array}\right)$ -- $3$ вектора уложенные по строчкам - это базис в $U$
        \\\\
        Теперь нужно проверить, можно ли выразить $v_1$ и $v_2$ через базис $U$, то есть по сути, проверить, лежат ли они в $U$. Для этого уложим их по строчкам в полученную на предыдущем шаге матрицу, и с помощью элементарных преобразований $1-$го типа попробуем занулить их:
        \\\\
        $\left(\begin{array}{rrrr}
        3 & -2 & -3 &   2\\
        0 &  1 &-15 &  11\\
        0 &  0 & 11 & 1\\
        -6 & -5 & 6 & 2\\
        -17 & 13 & -8 & 7\\
        \end{array}\right) \overset{\elon{4}{1}{2}}{\underset{\elon{5}{1}{17}}{\arrth{5}{3}}}
        \left(\begin{array}{rrrr}
        3 & -2 & -3 &   2\\
        0 &  1 &-15 &  11\\
        0 &  0 & 11 & 1\\
        0 & -9 & 0 & 6\\
        0 &  5 & -75 & 55\\
        \end{array}\right) \overset{\elon{5}{2}{5}}{\underset{\elon{4}{2}{3}}{\arrth{4}{\frac{1}{3}}}}
        \left(\begin{array}{rrrr}
        3 & -2 & -3 &   2\\
        0 &  1 &-15 &  11\\
        0 &  0 & 11 &  1\\
        0 &  0 &-45  & 35\\
        0 &  0 & 0  &  0\\
        \end{array}\right) \arrth{4}{\frac{1}{5}}
        \\\\\\
        \left(\begin{array}{rrrr}
        3 & -2 & -3 &   2\\
        0 &  1 &-15 &  11\\
        0 &  0 & 11 &   1\\
        0 &  0 & -9 &   7\\
        0 &  0 &  0 &   0\\
        \end{array}\right) \overset{\elth{4}{11}}{\arron{4}{3}{9}}
        \left(\begin{array}{rrrr}
        3 & -2 & -3 &   2\\
        0 &  1 &-15 &  11\\
        0 &  0 & 11 &   1\\
        0 &  0 & 0 &   68\\
        0 &  0 &  0 &   0\\
        \end{array}\right) \arrth{4}{\frac{1}{68}}
        \left(\begin{array}{rrrr}
        3 & -2 & -3 &   2\\
        0 &  1 &-15 &  11\\
        0 &  0 & 11 &   1\\
        0 &  0 &  0 &   1\\
        0 &  0 &  0 &   0\\
        \end{array}\right)$
        \\\\
        Итак: $v_1$ не лежит в $U$, т.к. $v_1$ не лежит в множестве векторов порождаемых каким-то базисом $U$, следовательно не лежит в $U$, $v_2$ лежит в каком-то базисе $U$, следовательно лежит в $U$. То есть только один из данных двух векторов лежит в $U$. $ЧТД$.
        \\
        Теперь дополним его $(v_2)$ до базиса $U$. Сделаем это следующим образом: у нас сейчас есть базис $U$ состоящий из векторов:\\
        $x_1 = \left(\begin{array}{rrrr} 3 & -2 & -3 & 2\\ \end{array}\right)$\\\\
        $x_2 = \left(\begin{array}{rrrr} 0 & 1 & -15 & 11\\ \end{array}\right)$\\\\
        $x_3 = \left(\begin{array}{rrrr} 0 & 0 & 11 & 1\\ \end{array}\right)$\\\\
        При этом, $v_2 \cdot 3 = x_1 \cdot 17 + x_2 \cdot 5$, следовательно т.к. $x_1$ и $x_2$ -- линейно независимы,$v_2$ и $x_2$ так же линейно независимы, следовательно $v_2$, $x_2$, $x_3$ - базис $U$.\\

    \section*{Задача \textnumero 3}
        Для решения данной чудесной задачи мне нужен очередной стандартный алгоритм: найдём какую-нибудь ФСР для ОСЛУ $A = (u_1, u_2, u_3, u_4), \quad Ax = 0$, уложим полученые векторы в матрицу по строкам, и составим ОСЛУ для этой матрицы. Полученная ОСЛУ и будет ответом на задачу. (Видимо нужно сказать магические слова про алгоритм с семинаров)
        \\\\
        $\left(\begin{array}{rrrr}
        4 & -5 & -7 & -4\\
       -1 &  5 & -2 & 16\\
       -1 & -1 &  4 & -8\\
        2 &  3 & -9 & 20\\    
        \end{array}\right) \arrtw{1}{3}
        \left(\begin{array}{rrrr}
       -1 & -1 &  4 & -8\\
       -1 &  5 & -2 & 16\\
        4 & -5 & -7 & -4\\
        2 &  3 & -9 & 20\\    
        \end{array}\right) \overset{\elon{2}{1}{-1}}{\underset{\elon{3}{1}{4}}{\arron{4}{1}{2}}}
        \left(\begin{array}{rrrr}
       -1 & -1 &  4 &  -8\\
        0 &  6 & -6 &  24\\
        0 & -9 &  9 & -36\\
        0 &  1 & -1 &   4\\    
        \end{array}\right) \overset{\elth{2}{\frac{1}{3}}}{\underset{\elth{3}{-\frac{1}{3}}}{\arrth{1}{-1}}}
        \\\\\\
        \left(\begin{array}{rrrr}
        1 &  1 & -4 &  8\\
        0 &  1 & -1 &  4\\
        0 &  1 & -1 &  4\\
        0 &  1 & -1 &  4\\    
        \end{array}\right) \overset{\elon{3}{2}{-1}}{\underset{\elon{4}{2}{-1}}{\arron{1}{2}{-1}}}
        \left(\begin{array}{rrrr}
        1 &  0 &  3 &  4\\
        0 &  1 & -1 &  4\\
        0 &  0 &  0 &  0\\
        0 &  0 &  0 &  0\\    
        \end{array}\right)$
        \\\\
        Итак, можно выбрать следующую ФСР для этой системы:\\\\
        $v_1 = \left(\begin{array}{r} -3\\ 1\\ 1\\ 0\\ \end{array}\right)$
        $v_2 = \left(\begin{array}{r} -4\\ -4\\ 0\\ 1\\ \end{array}\right)$\\\\
        Тогда возможный ответ выглядит так:
        $\left(\begin{array}{rrrr}
        -3 &   1 & 1 &  0\\
        -4 &  -4 & 0 &  1\\
        \end{array}\right)$

    \section*{Задача \textnumero 4}
        Вот это я понимаю, в первых трёх задачах нужно сделать $3$ Гаусса и в четвёртой ещё $3$, мда...\\
        План решения: выделить базисы в $L_1$ и $L_2$, а потом сделать чёрную магию. Выделять будем какие-то базисы, то есть уложим векторы в матрицу по строкам и сделаем прямого Гаусса, те векторы, которые не занулятся и есть искомый базис. (Да, я ещё раз скажу магические слова про алгоритм с семинаров и что ты мне сделаешь я в другом городе)
        \\\\
        Выделим базис в $L_1:$\\\\
        $\left(\begin{array}{rrrr}
        1 & -21 & 3 & 1\\
         4 & 3 & -3 & 1\\
      -6 & -19 & 7 & -1\\
        3 & -5 & -1 & 1\\
        \end{array}\right) \overset{\elon{2}{1}{-4}}{\underset{\elon{3}{1}{6}}{\arron{4}{1}{-3}}}
        \left(\begin{array}{rrrr}
        1 & -21 & 3 & 1\\
        0 & 87 & -15 & -3\\
        0 & -145 & 25 & 5\\
        0 &  58 & -10 & -2\\
        \end{array}\right) \overset{\elth{2}{\frac{1}{3}}}{\underset{\elth{3}{\frac{1}{5}}}{\arrth{5}{\frac{1}{2}}}}
        \\\\\\
        \left(\begin{array}{rrrr}
        1 & -21 & 3 & 1\\
        0 & 29 & -5 & -1\\
        0 & -29 & 5 & 1\\
        0 &  29 & -5 & -1\\
        \end{array}\right) \overset{\elon{3}{2}{1}}{\arron{4}{2}{-1}}
        \left(\begin{array}{rrrr}
        1 & -21 & 3 & 1\\
        0 & 29 & -5 & -1\\
        0 & 0 & 0 & 0\\
        0 & 0 & 0 & 0\\
        \end{array}\right)$ 
        \\\\
        Следовательно $dim(L_1) = rk(L_1)=2$, базис состоит из двух векторов:\\
        $v_1 = \left(\begin{array}{rrrr} 1 & -21 & 3 & 1\\ \end{array}\right)$ \\
        $v_2 = \left(\begin{array}{rrrr} 0 & 29 & -5 & -1\\ \end{array}\right) \\$
        \\
        Выделим базис в $L_2:$
        \\\\
        $\left(\begin{array}{rrrr}
        4 & 5 & -3 & -1\\
        3 & -4 & -1 & 0\\
        2 & -13 & 1 & 1\\
        -1 & 22 & -3 & -2\\
        \end{array}\right) \arrtw{1}{4}
        \left(\begin{array}{rrrr}
        -1 & 22 & -3 & -2\\
        3 & -4 & -1 & 0\\
        2 & -13 & 1 & 1\\
        4 & 5 & -3 & -1\\
        \end{array}\right) \overset{\elon{2}{1}{3}}{\underset{\elon{3}{1}{2}}{\arron{4}{1}{4}}}
        \\\\\\
        \left(\begin{array}{rrrr}
        -1 & 22 & -3 & -2\\
         0 & 62 & -10 & -6\\
         0 & 31 & -5 & -3\\
         0 & 93 & -15 & -9\\
        \end{array}\right) \overset{\elth{2}{\frac{1}{2}}}{\arrth{4}{\frac{1}{3}}}
        \left(\begin{array}{rrrr}
        -1 & 22 & -3 & -2\\
         0 & 31 & -5 & -3\\
         0 & 31 & -5 & -3\\
         0 & 31 & -5 & -3\\
        \end{array}\right)\overset{\elon{3}{2}{-1}}{\arron{4}{2}{-1}}
        \left(\begin{array}{rrrr}
        -1 & 22 & -3 & -2\\
         0 & 31 & -5 & -3\\
         0 & 0 & 0 & 0\\
         0 & 0 & 0 & 0\\
        \end{array}\right)$
        \\\\
        Следовательно $dim(L_2) = rk(L_2)=2$, базис состоит из двух векторов:\\
        $u_1 = \left(\begin{array}{rrrr} -1 & 22 & -3 & -2\\ \end{array}\right)$ \\
        $u_2 = \left(\begin{array}{rrrr} 0 & 31 & -5 & -3\\ \end{array}\right) \\$
        \\
        Теперь чёрная магия будет выглядеть следующим образом: чтобы выделить базисы $L_1 + L_2$ и $L_1 
        \cap L_2$ уложим в одну общую матрицу базисы $L_1$ и $L_2$ по строкам, а затем, с помощью элементарных преобразований первого и третьего типа доприведём часть матрицы заданную базисом $L_2$ к ступенчатому виду. Все векторы, которые получется таким образом занулить (нужно учесть, что занулённый вектор - это не тот, который записан в строчке изначально, а линейная комбинация $u_i$-х, которую занулили линейной комбинацией $v_i$-х, и нужно брать в качестве ответа именно эту линейную комбинацию) образуют базис $L_1 \cap L_2$, т.к. лежат и в $L_1$ и в $L_2$, а оставшиеся дополняют базис $L_1$ до базиса $L_1 + L_2$ (это даже не магия, а стандартный алгоритм с семинаров!)\\
        P.S. Если это не достаточно подробное объяснение с твоей точки зрения, я готов забиваться.
        \\\\
        $\left(\begin{array}{rrrr}
        1 & -21 & 3 & 1\\
        0 & 29 & -5 & -1\\
        -1 & 22 & -3 & -2\\
         0 & 31 & -5 & -3\\
        \end{array}\right) \overset{\elon{3}{1}{1}}{\arrth{3}{29}}
        \left(\begin{array}{rrrr}
        1 & -21 & 3 & 1\\
        0 & 29 & -5 & -1\\
        0 & 29 & 0 & -29\\
        0 & 31 & -5 & -3\\
        \end{array}\right) \arron{3}{2}{-1}
        \left(\begin{array}{rrrr}
        1 & -21 & 3 & 1\\
        0 & 29 & -5 & -1\\
        0 & 0 & 5 & -28\\
         0 & 31 & -5 & -3\\
        \end{array}\right) \arrth{4}{29}
        \\\\
        \left(\begin{array}{rrrr}
        1 & -21 & 3 & 1\\
        0 & 29 & -5 & -1\\
        0 & 0 & 5 & -28\\
        0 & 899 & -145 & -87\\
        \end{array}\right) \arron{4}{2}{-31}
        \left(\begin{array}{rrrr}
        1 & -21 & 3 & 1\\
        0 & 29 & -5 & -1\\
        0 & 0 & 5 & -28\\
        0 & 0 & 10 & -56\\
        \end{array}\right) \arron{4}{3}{-2}
        \left(\begin{array}{rrrr}
        1 & -21 & 3 & 1\\
        0 & 29 & -5 & -1\\
        0 & 0 & 5 & -28\\
        0 & 0 & 0 & 0\\
        \end{array}\right)$
        \\\\
        То есть согласно чёрной магии, $dim(L_1 + L_2) = rk(L_1 + L_2) = 3$, и можно взять в нём вот такой базис:\\
        $sum_1 = \left(\begin{array}{rrrr} 1 & -21 & 3 & 1\\\end{array}\right)$\\
        $sum_2 = \left(\begin{array}{rrrr} 0 & 29 & -5 & -1\\\end{array}\right)$\\
        $sum_3 = \left(\begin{array}{rrrr} 0 & 0 & 5 & -28\\\end{array}\right)$\\
        А так же, согласно чёрной магии, $dim(L_1 \cap L_2) = 1$ (это не совсем магия, т.к. следует из тождества $dim(A) + dim(B) = dim(A + B) + dim(A \cup B)$), и можно взять в этом пространстве базис следующим образом: рассмотрим линейную комбинацию векторов которой мы занулили нижнюю строчку\\
        $u_1 \cdot 29 - v_2 \cdot 31 - 2 \cdot ((u_1 + v_1) \cdot 29 - v_2) = 0$\\
        $u_1 \cdot 29 - v_2 \cdot 31 - 2 \cdot (u_1 \cdot 29 + v_1 \cdot 29 - v_2) = 0$\\
        $u_1 \cdot 29 - v_2 \cdot 31 - 2 \cdot 29 \cdot u_1 - 2 \cdot 29 \cdot v_1 + 2 \cdot v_2 = 0$\\
        $u_1 \cdot 29 - v_2 \cdot 29 - 58 \cdot u_1 - 58 \cdot v_1  = 0$\\
        $-58 \cdot v_1 - 29 \cdot v_2 = 58 \cdot u_1 - 29 \cdot u_1$\\
        $-2 \cdot v_1 - v_2 = 2 \cdot u_1 - u_1$\\
        Заметим, что мы нашли то, что требуется: $dim(L_1 \cap L_2) = 1$, и мы нашли вектор, такой, что он выражается и через базис $L_1$ и через базис $L_2$, следовательно является базисным для пространства $L_1 \cap L_2$. \\
        Вот он в числах:\\
        $inter_1 = \left(\begin{array}{rrrr} -2 & 13 & -1 & -1 \end{array}\right)$\\
        % $inter_1 = \left(\begin{array}{rrrr} 0 & 31 & -5 & -3\\ \end{array}\right) \\$
        % (v_1, v_2, u_1, u_1) -> (v_1, v_2, (u_1 + v_1) * 29, u_1) -> (v_1, v_2, (u_1 + v_1) * 29 - v_2, u_1) -> (v_1, v_2, (u_1 + v_1) * 29 - v_2, u_1 * 29) -> (v_1, v_2, (u_1 + v_1) * 29 - v_2, u_1 * 29 - v_2 * 31) -> (v_1, v_2, (u_1 + v_1) * 29 - v_2, u_1 * 29 - v_2 * 31 - 2 * ((u_1 + v_1) * 29 - v_2))
        
\end{document}